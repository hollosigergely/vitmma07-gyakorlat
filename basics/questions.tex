\section{Kérdések}\label{sec:questions}
Az ellenőrző kérdések során a két csoportból 1-1 szabadon választott kérdésre kell a választ elküldeni a 3--3 pluszpontért.

\subsection{Helymeghatározás alapjai}
\begin{itemize}
	\item Milyen viszonyítási rendszereket ismerünk a helymeghatározása témakörében?
	\item Írja le egy $\phi$ szöggel történő, tetszőleges $u=\left[u_x,u_y,u_z\right]$ tengely körüli forgatást jelképező kvaterniót!
	\item Írja le, miként számítjuk ki egy Rodrigues-vektorhoz tartozó forgatási mátrixot!
\end{itemize}

\subsection{Helymeghatározási feladat megoldása}
\begin{itemize}
	\item Ha $x$ jelöli a mérési vektorunkat, $f$ a modellünket és $\theta$ a paraméterhalmazt, írja fel a legkisebb négyzetes eltérés költségfüggvényét!
	\item Írja fel a Newton-Gauss nemlineáris egyenletek megoldási módszerének frissítési egyenletét!
	\item Írja fel a Bayes tételt! Jelölje a prior, posterior és hipotézis fogalmakat!
	\item Írja fel a kiterjesztett Kalman-szűrő állapotegyenleteit!
	\item Írja fel a RANSAC valószínűségi megállási feltétételét!
\end{itemize}