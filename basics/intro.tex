\section{Bevezetés}
A \emph{Navigációs szolgáltatások és alkalmazások} tantárgy \emph{Helymeghatározás alapjai} gyakorlat két fő részből áll, az első részben a hely és orientációval kapcsolatos feladatok megoldása kerül terítékre, míg a második részben a helymeghatározási feladat megoldását végezzük el.

A számítások során vagy kézi erőt alkalmazunk, vagy a Matlab programot használjuk. A Matlab kiváltható az ingyenes Octave\footnote{\url{https://www.gnu.org/software/octave/index}} programcsomaggal.